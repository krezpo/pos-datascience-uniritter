A Acácia Saúde é uma das principais redes hospitalares do país. Assim como grande parte dos hospitais, a Acácia Saúde sofre constantemente com o desperdício de alimentos. Com o intuito de reduzir os custos futuros com desperdício de alimentos, podemos inferir que este é um problema de previsão, \textit{i.e., forecast}. Problemas deste nicho podem ser resolvidos com regressões, uma subárea de aprendizado supervisionado \cite{caruana2006empirical}. Sendo mais específico, para o caso em questão, podemos aplicar regressão linear múltipla \cite{aiken2012multiple}. Conforme demonstrado no vídeo de apresentação, consideremos que os dados necessários para resolução do caso já foram unificados em uma única tabela estruturada. Assim, podemos utilizar duas variáveis independentes (a Média de Pedidos por Leito em \textit{kg} e o Desperdício Médio por Leito em \textit{kg}) e subtraí-las, para assim criar uma variável independente (Pedido Ideal em \textit{kg}). Com os dados dos anos de 2017, 2018 e 2019, podemos treinar um algoritmo de Regressão Linear, ofertado em bibliotecas como a \textit{Scikit-Learn} \cite{scikit-learn}. Após a fase de treinamento deste algoritmo, podemos realizar a previsão de novos anos, levando em consideração as variáveis independentes e a variável dependente como alvo, que em nosso caso é o (Pedido Ideal em \textit{kg}) para cada leito. Deste modo, podemos concluir que, mesmo com a inflação percebida no período é possível prever a quantidade ideal de alimentos que deverá ser solicitada pela rede hospitalar. Concluímos que, com a adoção de \textit{Machine Learning}, este ato poderá diminuir o custo operacional da Rede Acácia Saúde com desperdício de alimentos.