%Introduzir
A Alimentos LTDA é uma das maiores empresas do ramo alimentício Brasil. A Alimentos LTDA está na busca de agilidade, qualidade e economia nos projetos de inovação, de produtos e processos, em que cada vez mais orçamentos são enxutos, e informações instantâneas, seguras e de qualidade são essenciais para a tomada de decisão. Contudo, a empresa não consegue visualizar cenários prospectivos, tendo em vista a instabilidade econômica e de saúde no mundo. Diante desse cenário, são necessários grande volume, variedade, velocidade e veracidade de dados, que devem ser ingeridos pela organização. Como especialista em arquitetura e modelos de dados da Alimentos LTDA, tenho como responsabilidade auxiliar na construção de uma estratégia que contribua com o objetivo da empresa.

% Apresente, em um parágrafo, a descrição das ações estratégicas de Ingestão de Dados para a melhoria do negócio.

Para resolver este problema, devemos em um primeiro momento, elaborar ações estratégicas para auxiliar o negócio através da ingestão de dados em um \textit{datalake} \cite{khine2018data}. Um \textit{datalake} é um repositório de dados centralizado, que permite a ingestão de dados estruturados, semi-estruturados e não estruturados, de qualquer escala \cite{khine2018data}. A primeira ação para inserir dados em um \textit{datalake} é mapear as fontes de dados da empresa, além de seus respectivos tipos de dados.

%falar da ELK
A cada dia surgem novas ferramentas com grande potencial para resolução de problemas em ambientes \textit{big data} e \textit{fast data} \cite{miloslavskaya2016big}. Após um tempo entendendo o cenário da Alimentos LTDA, propomos a utilização de uma \textit{stack} que vem ganhando bastante notoriedade nos últimos anos, a ELK Stack \cite{chhajed2015learning}. Esta \textit{stack} nasceu com o ElasticSearch, onde o foco era a busca e análise de dados distribuídos. Desenvolvido sobre o Apache Lucene \cite{lucene2010apache}, a Elastic incorporou novos componentes ao seu produto, e o renomeou para ELK Stack. Atualmente, a ELK Stack possui componentes que facilitam a tanto a ingestão de dados, quanto o processo de ETL \cite{vassiliadis2002conceptual}, através dos componentes Logstash e Beats. 

%Falar dos componentes
A ELK Stack se tornou popular por possuir APIs de simples utilização. Além disso, esta \textit{stack} foi concebida para atuar de forma escalável, veloz e distribuída \cite{son2017performance}. Apesar do ElasticSearch ser o componente principal da ELK Stack, ela também possui ferramentas gratuitas e \textit{open-source} para ingestão, enriquecimento, armazenamento, análise e visualização de dados. Os componentes da ELK Stack são:

\begin{itemize}
    \item \textbf{ElasticSearch}: Mecanismo de armazenamento e busca que utiliza uma estrutura de dados baseado em índice invertido, projetada para buscas de texto rápidas;
    
    \item \textbf{Logstash}: Pipeline para ingestão de dados do lado do servidor. O Logstash pode receber, tratar e enriquecer dados de diversas fontes;
    
    \item \textbf{Kibana}: Ferramenta da ELK Stack responsável pela visualização e gerenciamento dos dados. Com o Kibana é possível criar histogramas, mapas e diferentes tipos de gráficos com base nos dados armazenados no ElasticSearch;
    
    \item \textbf{Beats}: Por fim, os Beats são microagentes de coleta de dados que ficam em aplicações clientes. Os Beats podem enviar dados de diversos dispositivos ou sistemas para o Logstash ou ElasticSearch.
    
\end{itemize}

%Concluir

A ELK Stack é uma boa opção para a Alimentos LTDA atingir seu objetivo, pois possui ferramentas especificas para coleta, enriquecimento, ingestão, análise, busca e visualização dos dados. Por sua natureza distribuída, a ELK Stack utiliza \textit{sharding} para particionamento horizontal, podendo trabalhar em \textit{clusters}, o que garante a resiliência das informações armazenadas \cite{kononenko2014mining}. Por fim, adotar esta \textit{stack} tecnológica pode ajudar a Alimentos LTDA a atingir seu objetivo de ter dados em real-time para auxiliar na tomada de decisão \textit{data-driven} e, se manter como uma das principais empresas do ramo alimentício nacional.