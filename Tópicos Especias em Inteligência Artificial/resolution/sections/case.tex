% intro
A Gazu é uma maiores fabricantes nacionais de automóveis. Presidida por Sales de Carvalho, a Gazu tem como objetivo inovar no setor automobilístico. Sales está motivado com as vantagens de mercado que carros autônomos podem trazer, tanto para a Gazu, quanto para os proprietários de seus veículos. Entretanto, carros autônomos são sistemas críticos, uma vez que qualquer falha pode acarretar em prejuízos financeiros e até mesmo vítimas fatais \cite{efing2020uso}. Por este motivo, Sales de Carvalho nos contratou para investigar como podemos garantir que os carros autônomos da Gazu se comportem de maneira adequada nas mais diversas ocasiões de risco no transito.

% contextualizar sobre carros autonomos

A tecnologia vem evoluindo constantemente, permitindo que a Inteligência Artificial seja aplicada nos mais diversos campos. Assim como o entretenimento, setor bancário, varejo, entre outras áreas, o setor automobilístico também está sendo favorecido com este progresso. Apesar de não ser algo recente, carros autônomos vem crescendo em popularidade e ajudando a evitar que vidas sejam perdidas \cite{caetano2020algoritmos}. Carros autônomos são automóveis que conseguem interpretar situações e tomar decisões sozinhos, sem a ajuda de um humano. Em outras palavras, carros autônomos são capazes de dirigir sozinhos \cite{de2018futuro}.

É importante destacar que carros autônomos são dotados de diversos sensores. Estes sensores são responsáveis por coletar informações sobre o ambiente onde o veículo se encontra. Além dos sensores, este tipo de veículo também possui atuadores para controlar o veículo de acordo com a necessidade. Estes equipamentos precisam ter uma rede rápida e confiável, para que os dados cheguem da melhor maneira nas centrais de processamento. As centrais de processamento são dotadas de algoritmos baseados em aprendizado de máquina \cite{trindade2018detecccao}. 

Devido os diversos eventos que podem acontecer no trânsito, os algoritmos de carros autônomos necessitam ser bem treinados e muito bem calibrados \cite{caetano2020algoritmos}. Por este motivo, uma equipe de profissionais qualificados é imprescindível para projetos deste tipo. Além do desenvolvimento, os testes do veículos devem ser rigorosos, pois somente com testes dos mais diversos, ter mais confiança no funcionamento dos carros autônomos. Seguindo essas recomendações, a Gazu terá sucesso no desenvolvimento e nas vendas de seus carros autônomos.