Grande parte da população do município de Winderson sofre de glicêmia alterada, uma condição crônica. Preocupado com a saúde da população, o município organiza constantemente palestras e eventos para educar a população sobre os cuidados com a saúde, uma vez que a glicêmia elevada pode causar doenças como a diabetes. Atualmente, os prontuários dos cidadãos, assim como materiais dos eventos realizados no município são armazenados em planilhas. Organizar muitas informações em planilhas pode tornar o processo de recuperação da informação muito complexo, moroso e não-escalável. Nosso analista de sistemas tem a oportunidade de aplicar ferramentas e conceitos de Web Semântica \cite{berners2001semantic} para melhorar o controle das informações médicas dos munícipes. Para atingir seu objetivo, o nosso analista poderá utilizar dois \textit{schemas} amplamente conhecidos, como:

\begin{itemize}
    \item A \textit{Friend of a Friend} \cite{graves2007foaf} (FOAF), uma vez que a diabetes pode ser considerada uma condição hereditária, tornando assim interessante saber a relação familiar entre as pessoas;
    
    \item E, a \textit{Dublin Core} \cite{baker2000grammar} (DC), que, por sua vez, poderá armazenar os prontuários dos cidadãos, assim como os materiais dos eventos realizados no município de Winderson.
\end{itemize}
    
O analista poderá também utilizar ferramentas como o \textit{Protégé} \cite{gennari2003evolution}. O \textit{Protégé} permite a criação, edição e manipulação de ontologias computacionais. Com o \textit{schema} definido, as ferramentas devidamente selecionadas, cabe ao nosso também desenvolver um sistema amigável, \textit{web} ou \textit{mobile}, que permita que os agentes de saúde de Winderson insiram informações sobre os os munícipes e seus prontuários, além dos materiais sobre o eventos organizados na cidade. Tal sistema deverá possibilitar aos agentes de saúde pesquisar informações sobre os cidadãos do município, de modo que seja possível acompanhar se os eventos realizados estão surtindo efeito na saúde da população. O sistema deverá converter as consultas dos agentes de saúde em \textit{queries} SPARQL \cite{perez2009semantics}, para que seja possível retornar informações da ontologia adotada em Winderson. Com esta solução baseada em Web Semântica, os agentes de saúde do município de Winderson poderão fazer consultas com maior agilidade e acompanhar o impacto das ações realizadas na saúde das pessoas, trazendo incontáveis benefícios para a população.